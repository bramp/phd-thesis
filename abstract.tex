

%\begin{abstract}
%specific program, how I approached the problem,
%not focused enough, What I've done, why, what I got out of it
%what I've done, what I've achieved, and why? Make it exciting
%If you don't understand the topic after the abstract, then you have problems
%If you read the introduction, and you still don't understand you have major problems
%must reflect my vision, and what I've done
%few bullet points, what is important, sentences then paragraph – top down
%what is important to be said
%gun to my head, 30seconds to impress them
%read some other thesis
%original and significant contribution
%\end{abstract}

%\begin{itemize}
%  \item Start-to-finish model isn't used.
%  \item High levels of interactive observed, aided by bookmarks
%  \item Not all of the video is equally popular, they are ranked based on power-law distributions.
%  \item Caused hotspots, which normally start at a bookmark
%  \item Techniques to dynamically move bookmark as well as work out the corresponding length
%  \item The high interactive loses us some predictability. However the order in which they view bookmarks is predictable
%  \item All of this has been modelled
%\end{itemize}

\phantomsection
\addcontentsline{toc}{chapter}{Abstract}

%\chapter{Abstract}
%\begin{abstract}
% No more than 300 words

\begin{center}
{\LARGE\bfseries \thetitle\\}
{\Large\bfseries \theauthor\\}
{\large Thesis submitted for the degree of Doctor of Philosophy\\
\thedate}
\end{center}

\section*{Abstract}

The traditional start-to-finish playback model is not suitable for all modern interactive video streams. Users expect support for higher levels of interactivity such as fast forward and rewind or the ability to arbitrary seek within their media quickly and efficiently. By conducting user studies we have observed start-to-finish is not applicable to many genres of video, and that different playback models fit better. We discuss how existing delivery techniques are impacted by these new observations.

Novel interactive controls such as bookmarks have also highly impacted user behaviour. This has lead to the segments within the media being accessed in a uneven fashion, causing hotspots of interest to form; areas with orders of magnitudes more viewers than others. These hotspots typically began at the beginning of a bookmark, however not always, which lead us to design a dynamic bookmark positioning algorithm. As well as their position, determining the hotspot's length can be beneficial. This aids in autonomic techniques such as replication and pre-fetching as well as allowing the users to find what they want quicker.

Under high level of interactivity, delivery techniques are less efficient due to the unpredictability of the users. We however developed techniques which restore some of this predictability, allowing clients or servers to predict future actions based on past user actions. These technique proves exceeding useful for pre-fetching which reduces seek latencies for client and can reduce load on servers. However knowledge of past user activities need to be gathered from network, thus we develop techniques to do this in a distributed manner.


%\section{The Pitch}

%Media streaming is increasingly being used online, with CDNs being deployed to handle the sheer volume of requests. As such systems are becoming commonplace, users are expecting improved functionality, for example interactive controls. This is increasingly difficult to provide with differing content types, and when existing CDN techniques do not work well under interactive workloads. I show that simple interactive workloads are exceedingly different from traditionally discovered workloads (such as the start-to-finish model) and that these new workloads add a degree of unpredictability. However by adding simple interactive features such as bookmarks, and observing past user behaviour, it is possible to gain some insight into what the user will watch, and in what order. Techniques have been developed to gather the past user actions in a distributed manner~\footnote{Distributed techniques will be developed soon} and to exploit this data to improve playback for all users under an interactive scenario.

%\end{abstract} 

\chapter*{Declaration}

\large

This thesis is a presentation of my original research work. No part of this thesis has been submitted
elsewhere for any other degree or qualification. All work is my own unless otherwise
stated. The work was carried out under the guidance of Laurent Mathy and Nicholas Race, at Lancaster University's Computing Department.

\vspace{3.5cm}

\begin{center}
\begin{tabular}{lll}
  \rule[-0.5em]{7cm}{0.5pt} & \hspace{1cm} & 30\sth September 2008  \\
  Andrew Brampton & & Date \\
\end{tabular}
\end{center}

\vspace{3.5cm}

\noindent
Copyright \copyright 2008 by Andrew Brampton.\\
``The copyright of this thesis rests with the author. No quotations should be published
or information and results derived from this thesis without acknowledgement.''
